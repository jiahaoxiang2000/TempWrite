\documentclass{beamer}
\usepackage{ctex} % To support Chinese characters, if needed
\usepackage{graphicx} % To include images
\usepackage{hyperref} % To include hyperlinks
\usepackage{amsmath} % For mathematical symbols
\usepackage{booktabs} % For beautiful tables

% Theme choice:
\usetheme{Madrid}
\usecolortheme{seahorse}

% Custom block colors
\setbeamercolor{block title}{bg=blue!30,fg=black}
\setbeamercolor{block body}{bg=blue!10,fg=black}
% \setbeamercolor{alertblock title}{bg=red!50,fg=white}
% \setbeamercolor{alertblock body}{bg=red!20,fg=black}
\setbeamercolor{exampleblock title}{bg=green!50,fg=black}
\setbeamercolor{exampleblock body}{bg=green!20,fg=black}

% Enable figure numbering
\setbeamertemplate{caption}[numbered]


% Title, author, and date information:
\title{\textbf{Windows to MacOS}}
\author[xjh]{Jiahao Xiang}

\date{\today}

\begin{document}

\begin{frame}
    \titlepage
\end{frame}

\begin{frame}
    \frametitle{Preface}
    
   \textbf{Start Point:} 前些日子,苹果出了Mac mini 4, 内存白给8G,配上活动打折,我们就入手了一台。 
   \vfill
    
   \textbf{Background:} Windows 15年老用户从xp到11,如果说用windows打打游戏,肯定没啥问题,但是出于coding的目的,UNIX类的系统更加适合。之前用过一段时间linux下的archlinux,但是linux下图形栈的问题,使用起来不舒服,很多大的或小的兼容性问题。

   \vfill
    \textbf{Goal:} 本文主要是macOS给我带来的一些新的体验。(个人体验,不一定适用于所有人,仅供参考)

\end{frame}


\begin{frame}
    \frametitle{System}
    \begin{block}{Desktop Environment}
        \begin{itemize}
            \item \textbf{Dock:} 任务栏,很少用默认隐藏
            \item \textbf{Mission Control:} 多桌面管理工具, 和window下近似,但是应用之间的切换逻辑不一致,需要适应
        \end{itemize}
    \end{block}
    \begin{block}{Devices Ecosystem}
        \begin{itemize}
            \item \textbf{IPhone:} 所有信息都可以在mac上同步,比如短信,电话,照片等
            \item \textbf{AirPods:} 无缝连接,可以在mac上直接切换音频输出设备  
        \end{itemize}
    \end{block}
\end{frame}

\begin{frame}
    \frametitle{GUI Applications}
    \begin{block}{Tools}
        \begin{itemize}
            \item \textbf{Alfred:} 一个很强大的应用,可以替代spotlight,可以自定义很多功能,比如搜索,快捷键等
            \item \textbf{Rectangle:} 一个窗口管理工具,可以用快捷键调整窗口大小,位置,符合windows下的习惯
        \end{itemize}
    \end{block}
    \begin{block}{Productivity}
        \begin{itemize}
            \item \textbf{Screen Studio:} mac下录屏工具,可以录制屏幕,摄像头,声音等
            \item \textbf{Wechat:} 微信,mac下的微信,功能比windows似乎还多,其他的社交软件也有mac版本,比如qq,钉钉等
        \end{itemize}
    \end{block}

    \begin{alertblock}{Conclusion}
        \textbf{macOS下的GUI应用生态很好(除了那些游戏、建模的软件深度绑定windows),有一部分应用是mac专用的。}
    \end{alertblock}
\end{frame}

\begin{frame}
    \frametitle{NO GUI Applications}
    \begin{block}{Homebrew}
        \begin{itemize}
            \item \textbf{Homebrew:} macOS下的包管理工具,可以安装很多软件,比如git,python,nodejs等
            \item \textbf{Homebrew Cask:} 一个扩展,可以安装一些GUI软件,比如vscode,firefox等
        \end{itemize}
    \end{block}
    \begin{alertblock}{Conclusion}
        \textbf{macOS下的命令行工具生态也很好,和linux下具有相同的开发体验,各种代码库都能方便集成。}
        
    \end{alertblock}
\end{frame}

\end{document}