\documentclass{beamer}
\usepackage{ctex}
% usepackage
\usepackage{url}

% Theme choice:
\usetheme{Madrid}


% Title page details: 
\title{\textit{轻量级密码学算法库}}
% \subtitle{A Personal Perspective}
\author[xjh]{向嘉豪\inst{1}}
\institute{
    \inst{1}
    衡阳师范学院
}
\date{\today}

\begin{document}

% Title page frame
\begin{frame}
    \titlepage
\end{frame}


% Outline frame
% \begin{frame}{Content}
%     \tableofcontents
% \end{frame}
\AtBeginSection[ ]
{
    \begin{frame}{\textit{目录}}
        \tableofcontents[currentsection]
    \end{frame}
}

% main content

\section{背景}

\begin{frame}{\textit{密码学算法库}}
    \begin{itemize}
        \item 特性
              \begin{itemize}
                  \item 数据的\textcolor{red}{机密性},对于攻击者来说不可见原始信息。
                  \item 数据的完整性,防止攻击者修改原始信息。
                  \item 数据的\textcolor{red}{可用性},实际应用中的易用性和有效性,如\textcolor{blue}{加密性能}。
                  \item 。。。
              \end{itemize}
        \item 上层通信协议
              \begin{itemize}
                  \item \textcolor{red}{SSL/TLS}:安全套接字层(SSL)和传输层安全(TLS),常用如HTTPS协议。
                  \item IPsec:互联网协议安全(IPsec)是一组协议,通常用于建立虚拟专用网络(VPN)。
                  \item SSH:安全外壳协议(SSH)是一种用于在不安全的网络上安全地访问远程计算机的协议。
                  \item PGP:非常好的隐私(PGP)是一种用于加密和解密数据的程序,通常用于保护电子邮件通信的安全。
              \end{itemize}
        \item \textcolor{blue}{我们每天都在使用这些协议,每天都在使用密码学算法库。}
    \end{itemize}
\end{frame}

\begin{frame}{\textit{轻量级密码学算法库}}
    \begin{itemize}
        \item \textcolor{red}{资源受限}的环境下,实现的密码学算法库。
              \begin{itemize}
                  \item 低功耗,如受限设备功耗为100mW,而PC机为250W。
                  \item 低存储,如受限设备存储为64KB,而PC机为1TB。
                  \item 低计算频率,如受限设备为64MHz,而PC机为5GHz。
              \end{itemize}
        \item 上层通信协议
              \begin{itemize}
                  \item \textcolor{red}{MQTT}: 轻量级的发布/订阅消息传输协议。例如,智能家居设备使用MQTT协议来传输传感器数据和控制命令。
                  \item CoAP: 受限制应用协议,近似与HTTP,常用于物联网(IoT)环境。例如,智能照明系统使用CoAP协议来控制灯光的开关和亮度。
              \end{itemize}
        \item \textcolor{blue}{如何在资源相差约1000倍的设备上实现密码学算法库?}
    \end{itemize}
\end{frame}

\section{痛点}

\begin{frame}{\textit{安全}}
    \begin{itemize}
        \item \textcolor{red}{安全}是密码学算法库最基本的要求。
              \begin{itemize}
                  \item 攻击者可以在获取设备的物理访问权限后,通过\textcolor{red}{侧信道攻击}(如功耗分析、时序分析、电磁分析等)来窃取设备中的敏感信息。
                  \item 利用协议漏洞,如\textcolor{red}{重放攻击}、\textcolor{red}{中间人攻击}、\textcolor{red}{拒绝服务攻击}等,干扰设备正常工作。
              \end{itemize}
        \item 如何在资源受限的设备上实现更\textcolor{blue}{安全}的密码学算法库?
    \end{itemize}
\end{frame}

\begin{frame}{\textit{性能}}
    \begin{itemize}
        \item \textcolor{red}{性能}限制加密库的使用场景。
              \begin{itemize}
                  \item 加密性能:加密速度、解密速度、加密延迟、解密延迟。
                  \item 存储性能:存储空间、存储延迟。
              \end{itemize}
        \item 如何在资源受限的设备上实现更\textcolor{blue}{高性能}的密码学算法库?
    \end{itemize}
\end{frame}

\begin{frame}{\textit{可移植性}}
    \begin{itemize}
        \item \textcolor{red}{可移植性}影响到加密库的可用性。
              \begin{itemize}
                  \item 跨平台:支持多种硬件平台,如ARM、MIPS、X86等。
                  \item 跨编程语言:支持多种编程语言,如C、C++、Python等。
              \end{itemize}
        \item 如何在资源受限的设备上实现更\textcolor{blue}{可移植}的密码学算法库?
    \end{itemize}
\end{frame}

\section{目标}

\begin{frame}{\textit{性能}}
    \begin{itemize}
        \item 使用预计算、查找表、位运算等\textcolor{blue}{通用优化技术},提高加密性能。
        \item 采用扩展指令集、硬件加速器等\textcolor{red}{特定优化技术},提高加密性能。
    \end{itemize}
\end{frame}

\begin{frame}{\textit{可移植性}}
    \begin{itemize}
        \item 使用\textcolor{blue}{标准化接口},如OpenSSL、mbedTLS等,提高可移植性。
        \item 使用\textcolor{blue}{模块化设计},提高可移植性。
        \item \textcolor{red}{多平台支持},如ARM、MIPS、X86等。
    \end{itemize}
\end{frame}


\begin{frame}{\textit{更安全}}
    \begin{itemize}
        \item 使用轻量级的密码原语,去\textcolor{red}{设计加密算法}。
        \item 设计抵抗侧信道攻击的\textcolor{red}{实现方案}。
        \item \textcolor{red}{修改通讯协议},增加安全性。
    \end{itemize}
\end{frame}


\begin{frame}{\textit{Thank You}}
    \centering
    \Large
    Thank you for your attention! \\
    \normalsize
    项目参考:\url{https://github.com/Tongsuo-Project/tongsuo-mini}.\\
\end{frame}

\end{document}