\documentclass{beamer}

% Theme choice:
\usetheme{Madrid}

% Title page details: 
\title{Writing a Paper}
% \subtitle{A Personal Perspective}
\author[xjh]{Jiahao Xiang\inst{1}}
\institute{
    \inst{1}
    Hengyang Normal University
}
\date{\today}

\begin{document}


% Title page frame
\begin{frame}
    \titlepage
\end{frame}

% Outline frame
% \begin{frame}{Content}
%     \tableofcontents
% \end{frame}
\AtBeginSection[ ]
{
    \begin{frame}{Content}
        \tableofcontents[currentsection]
    \end{frame}
}

% main content
\section{Prepare}
\begin{frame}{Viewer Perspective}
    \begin{itemize}
        \item \textbf{Interesting:} The main idea can be quickly understood, enticing the reader to continue.
        \item \textbf{Novel:} The reader can learn something new, indicating the amount of new knowledge provided.
        \item \textbf{Useful:} The results are high-quality, the methods are practical, and the conclusions are reasonable.
        \item \textbf{Well-written:} The paper is well-organized, the language is clear, and the grammar is correct.
    \end{itemize}
\end{frame}

\begin{frame}{Perfect Tools} % Start a new frame with the title "Perfect Tools"
    \begin{itemize} % Start an itemized list
        \item \textbf{Experiment Environment:} MATLAB, Python, etc. % Item with bold title
        \item \textbf{Reference Manager:} EndNote, Mendeley, etc. % Item with bold title
        \item \textbf{Writing Tools:} \LaTeX, Word, etc. % Item with bold title
        \item \textbf{Figure Tools:} Visio, Drawio, etc. % Item with bold title
        \item $\cdots$ % Item with ellipsis
    \end{itemize} % End of itemized list
\end{frame} % End of frame

\section{Writing}

\begin{frame}{Interesting}
    \begin{itemize}
        \item \textbf{Title:} The title should be concise, informative, and highlight the main differences.
        \item \textbf{Abstract:} The abstract should be clear and concise. It should primarily present the main idea.
    \end{itemize}
\end{frame}

\begin{frame}{Novel}
    \begin{itemize}
        \item \textbf{Introduction:} Present the research background and the main question that the paper addresses. Provide a concise description of the main idea.
        \item \textbf{Related Work:} The related work section should be comprehensive and detailed. Highlight the differences between your work and existing literature.
        \item \textbf{Method:} The method section should explain what to do and why to do it. The method should be clear and easy to understand.
    \end{itemize}
\end{frame}

\begin{frame}{Useful}
    \begin{itemize}
        \item \textbf{Experiment:} The experiment section should be detailed and reproducible. The results should compare the most recent and reliable data.
        \item \textbf{Discussion:} The discussion section should be concise, advantageous, and reasonable.
    \end{itemize}
\end{frame}

\begin{frame}{Well-written}
    \begin{itemize}
        \item \textbf{Grammar:} The grammar should be correct, and the language should be clear.
        \item \textbf{Organization:} The paper should be well-organized, with a clear structure and logical flow.
        \item \textbf{Figures:} The figures should be clear, informative, and well-labeled.
    \end{itemize}
\end{frame}

\section{Submission and Revision}

\begin{frame}{Submission}
    \begin{itemize}
        \item \textbf{Journal Selection:} Choose a journal that is suitable for your paper. consider the journal's scope, review time.
        \item \textbf{Submission System:} Follow the submission guidelines and submit your paper.
    \end{itemize}
\end{frame}

\begin{frame}{Revision}
    \begin{itemize}
        \item \textbf{Response Letter:} Write a response letter addressing each of the reviewers' comments individually. Ensure there's no confusion about the reviewers' comments and think carefully before responding.
        \item \textbf{Revision:} Revise your paper according to the reviewers' comments and highlight the changes you've made.
    \end{itemize}
\end{frame}

\begin{frame}{Thank You}
    \centering
    \Large
    Thank you for your attention! \\
\end{frame}

\end{document}