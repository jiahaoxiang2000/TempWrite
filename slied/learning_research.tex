\documentclass{beamer}
\usepackage{ctex} % To support Chinese characters, if needed
\usepackage{graphicx} % To include images
\usepackage{hyperref} % To include hyperlinks
\usepackage{amsmath} % For mathematical symbols
\usepackage{booktabs} % For beautiful tables

% Theme choice:
\usetheme{Madrid}
\usecolortheme{seahorse}

% Custom block colors
\setbeamercolor{block title}{bg=blue!30,fg=black}
\setbeamercolor{block body}{bg=blue!10,fg=black}
% \setbeamercolor{alertblock title}{bg=red!50,fg=white}
% \setbeamercolor{alertblock body}{bg=red!20,fg=black}
\setbeamercolor{exampleblock title}{bg=green!50,fg=black}
\setbeamercolor{exampleblock body}{bg=green!20,fg=black}


% Title, author, and date information:
\title{\textbf{Learning Research}}
\author[xjh]{Jiahao Xiang\inst{1}}
\institute{
    \inst{1}
    Hengyang Normal University
}
\date{\today}

\begin{document}

\begin{frame}
    \titlepage
\end{frame}

\begin{frame}
    \frametitle{Preface}
    % introduce the share speech motivation
    \textbf{Motivation:} 对于接触Research一年多,还是小白的我来说,不具备一套高效的方法论。这次分享,我们对常年混迹与顶会的,一位浙大的大佬(彭思达)分享的\texttt{learning\_research}进行学习,在大佬的输入下,输出一下我们学习到的内容,希望能够对大家也有所帮助。
    \vfill
    \begin{block}{大佬思想使用该颜色块标注}
        \url{https://github.com/pengsida/learning_research}
    \end{block}
    \begin{exampleblock}{我们的想法}
        我们汇报的slide: \url{https://github.com/jiahaoxiang2000/TempWrite/blob/master/slied/learning_research.pdf}
    \end{exampleblock}
    
    
\end{frame}

\begin{frame}
    \frametitle{Table of Contents}
    \tableofcontents
\end{frame}

\section{找问题}
\begin{frame}
    \frametitle{找问题}
    \begin{block}{一阶段}
        这个阶段追求\textcolor{blue}{广度},了解一些基础的概念和算法。不要求深度,不要求掌握/熟悉算法所有的细节。这个阶段的目的是让你对大方向有一个大概的了解,知道有哪些算法,知道这些算法的\textcolor{blue}{大概原理},知道这些算法的应用场景。
    \end{block}
    \begin{block}{二阶段}
        这个阶段追求\textcolor{blue}{深度},追求掌握某一篇论文的细节(算法细节、代码实现细节)。这个阶段的目标是构建某一个科研细分方向的算法基础,了解一篇\textcolor{red}{论文}是怎么做出来的(寻找科研问题、想idea、做实验、写论文)。
    \end{block}
    \begin{exampleblock}{找问题}
        当来到二阶段时,一类问题已经明显了,一类为旧的issue,我们阅读的文献;二类为新的issue,属于开创新的贡献。
    \end{exampleblock}

\end{frame}

\section{解问题}
\begin{frame}
    \frametitle{解问题}
    \begin{block}{三阶段}
        在有了一定算法基础以后,开始在实验室的指导下做一个自己一作的Project。这个阶段的目标是通过\textcolor{red}{实践}来学习一篇论文是怎么做出来的。
    \end{block}
    \begin{exampleblock}{想idea}
        想点子的过程,就是尝试去解问题的过程。找找旧的解法,看看有没有可以改进的地方,或者能不能引入一些新的思路。
    \end{exampleblock}
    \begin{block}{杨植麟认为}
        技术的本质就是对方法做\textcolor{red}{组合},把小的技术组合成大的技术,把老的技术组合成新的技术。
    \end{block}
\end{frame}

\begin{frame}
    \frametitle{想idea}
    具体的想idea的流程(Goal-driven research)
    \begin{block}{1. general goal}
        一般而言,general goal容易定义,但制定roadmap需要对领域有深刻的理解。可以通过构建literature tree来建立起对该领域的认知。
    \end{block}
    \begin{block}{literature tree}
        \begin{itemize}
            \item 收集相同方向的论文。
            \item 通过阅读论文,梳理出当前方向已有的milestone tasks,并标记提出该\textcolor{red}{task}的第一篇论文(1类novelty)。
            \item 将论文根据milestone tasks进行归类。梳理出有代表性的pipelines,并标记提出该\textcolor{red}{pipeline}的第一篇论文(2类novelty)。
            \item 根据pipeline再细分到novel \textcolor{blue}{module},归类论文(3类novelty)。加一些module改进已有pipeline地工作 (4类novelty)
            \item 随着自己对领域的理解,增加新的milestone tasks。
        \end{itemize}
    \end{block}
\end{frame}

\begin{frame}
    \frametitle{想idea}
    \begin{exampleblock}{novelty的分类}
        创新性越高,它所能影响的文章数量就越多。1类milestone task,2类novel pipeline,3类novel  module,4类旧module改进已有pipeline地工作。i.e. \textcolor{blue}{创新性很大程度上,影响文章录用的等级。}
    \end{exampleblock}

    
\end{frame}


\section{做实验}

\section{写论文}

\section{搓PPT}

\section{Conclusion}


\end{document}