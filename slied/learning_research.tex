\documentclass{beamer}
\usepackage{ctex} % To support Chinese characters, if needed
\usepackage{graphicx} % To include images
\usepackage{hyperref} % To include hyperlinks
\usepackage{amsmath} % For mathematical symbols
\usepackage{booktabs} % For beautiful tables

% Theme choice:
\usetheme{Madrid}
\usecolortheme{seahorse}

% Title, author, and date information:
\title{\textbf{Learning Research}}
\author[xjh]{Jiahao Xiang\inst{1}}
\institute{
    \inst{1}
    Hengyang Normal University
}
\date{\today}

\begin{document}

\begin{frame}
    \titlepage
\end{frame}

\begin{frame}
    \frametitle{Preface}
    % introduce the share speech motivation
    \textbf{Motivation:} 对于接触Research一年多的,还是小白的我来说,不具备一套高效的方法论。借助这次分享,我们对常年混迹与顶会的,一位浙大的大佬分享的\texttt{learning\_research}进行学习,在大佬的输入下,通过这次分享会,输出一下我们学习到的内容,希望能够对大家有所帮助。
    \vfill
    \href{https://github.com/pengsida/learning_research}{https://github.com/pengsida/learning\_research},我们汇报的slide:\href{URL}{text}
\end{frame}

\begin{frame}
    \frametitle{Table of Contents}
    \tableofcontents
\end{frame}

\section{Introduction}
\begin{frame}
    \frametitle{Introduction}
    \begin{itemize}
        \item Brief overview of the topic
        \item Importance of the research
        \item Objectives of the presentation
    \end{itemize}
\end{frame}

\section{Methodology}
\begin{frame}
    \frametitle{Methodology}
    \begin{itemize}
        \item Research methods used
        \item Data collection techniques
        \item Analysis approach
    \end{itemize}
\end{frame}

\section{Results}
\begin{frame}
    \frametitle{Results}
    \begin{itemize}
        \item Key findings
        \item Statistical results
        \item Interpretation of the results
    \end{itemize}
\end{frame}

\section{Conclusion}
\begin{frame}
    \frametitle{Conclusion}
    \begin{itemize}
        \item Summary of findings
        \item Implications of the research
        \item Future research directions
    \end{itemize}
\end{frame}

\end{document}