\documentclass[slide]{../custom}

\title{\textbf{复试指导-电子信息}}
\author{Isomo}

\begin{document}

\begin{frame}
  \titlepage
  \footnotesize
  \textcolor{blue}{中间有任何问题,可以随时打断,我会尽量解答。}
\end{frame}

% create the table of contents
\begin{frame}
  \frametitle{目录}
  \tableofcontents
\end{frame}

\section{主基调}

\begin{frame}
  \frametitle{主基调 - \textcolor{red}{保一志愿}}
  我们从近三年的招生名额、一志愿上线人数和录取人数来看,具体我们可以看表\ref{tab:招生数据}。22年和23年各有一个放弃复试的学生。24年调剂2批,分别为18,6。从近5年数据来看,其实一志愿上线\textcolor{blue}{有且只有一位同学}未录取,这位同学是23年的。(我们推论:\textcolor{red}{一志愿上线即上岸})
  \begin{table}[htbp]
    \centering
    \caption{招生数据(一志愿)}
    \begin{tabular}{cccc}
      \toprule
      年份 & 拟录取 & 一志愿上线人数 & 一志愿录取人数 \\
      \midrule
      2022 & 27 & 10 & 9 \\
      2023 & 32 & 34 & 32 \\
      2024 & 54 & 30 & 30 \\
      \bottomrule
    \end{tabular}
    \caption*{\footnotesize 注:上线人数和总报考人数关联较大,2025上线人数应小于30}
    \label{tab:招生数据}
  \end{table}

\end{frame}

\section{流程和Tips}

\begin{frame}
  \frametitle{流程}
  \begin{enumerate}
    \item \textbf{材料准备}:《硕士研究生招生复试、调剂、录取工作方案》\footnote{24年:\url{http://jkxy.hynu.cn/info/1017/7035.htm}},复试前一周公示。
    \item \textbf{笔试}:《软件工程》。NOTE:往年工科类专业不加考试。
    \item \textbf{复试}:先英文自我介绍,抽取英文提问,老师提问。
  \end{enumerate}
\end{frame}

\begin{frame}
  \frametitle{Tips}
  \begin{block}{材料准备}
    \begin{itemize}
      \item 身份证、准考证(原件+复印)
      \item 学历材料
        \begin{itemize}
          \item 应届:学生证、学籍在线验证报告(原件+复印)
          \item 往届:毕业证、学位证、学历备案表(原件+复印)
          \item 自考生:考籍卡、6科以上成绩单
          \item 境外学历:教育部认证报告(原件+复印)
          \item 专升本:另需专科毕业证(原件+复印)
        \end{itemize}
      \item 档案调令(非应届需要)
      \item 考生自述+能力证明(含获奖证书、论文等)
      \item 政审表(由档案所在单位/街道填写)
      \item 复试费120元缴费凭证
    \end{itemize}
  \end{block}
\end{frame}

\begin{frame}
  \frametitle{Tips}
  \begin{block}{笔试 - 软件工程 - \textcolor{blue}{抓住课后习题} }
    \begin{itemize}
      \item 软件工程基础知识
        \begin{itemize}
          \item 基本概念:软件特点、软件危机、软件工程原理
          \item 软件过程:生命周期、开发模型
          \item 标准化和发展趋势
        \end{itemize}
      \item 软件开发各阶段
        \begin{itemize}
          \item 需求分析与可行性研究
          \item 软件设计(总体设计、详细设计)
          \item 软件实现与测试
          \item 软件维护
        \end{itemize}
      \item 面向对象方法学
        \begin{itemize}
          \item 基本概念和UML建模
          \item 分析设计过程
          \item 软件重用和测试
        \end{itemize}
    \end{itemize}
  \end{block}
\end{frame}

\begin{frame}
  \frametitle{Tips}
  \begin{block}{复试 - 展示综合能力 - \textcolor{red}{沟通 + 技能}}
    \begin{itemize}
      \item 自我介绍 - 英文 (不能泄漏个人信息)
      \item 英文提问 - 非专业性简单问题
      \item \textcolor{red}{老师提问} - 专业性问题
        \begin{itemize}
          \item 专业基础知识
          \item 个人经历 - \textcolor{red}{项目、实习、竞赛}
          \item 未来规划
        \end{itemize}
    \end{itemize}
  \end{block}
  \begin{alertblock}{老师关心点}
    \begin{itemize}
      \item 能否毕业 - 需要发表\textcolor{red}{一篇SCI论文} - 技能
      \item 沟通成本 - 你能否融入团队
      \item \textcolor{red}{论文导向}逻辑:论文 $\rightarrow$ 课题,职称 $\rightarrow$ money  \\ 举例:JCR一区奖励6万,二区4万,XXX学院
    \end{itemize}
  \end{alertblock}
\end{frame}

\section{导师介绍}

\begin{frame}
  \frametitle{导师介绍 (1/2)}
  这里我们从导师的\textcolor{red}{科研方向}、\textcolor{red}{团队情况}(23级人数)和\textcolor{red}{导师性格}(此处为本人的接触和同学反应情况判定,\textcolor{red}{非准确})来看。

  \begin{table}[htbp]
    \centering
    \caption{导师介绍(按姓氏首字母排序)}
    \begin{tabular}{cccc}
      \toprule
      导师 & 科研方向 & 团队情况 & 导师性格 \\
      \midrule
      陈纪友 & 深度学习+图像 & -- & -- \\
      陈文辉 & 数据压缩算法? & 3 & 和蔼 \\
      陈中 & 图像加密 & 5 & 和蔼 \\
      邓红卫 & 深度学习+图像 & 3 & 认真 \\
      焦铬 & 深度学习+图像 & 7 & 和蔼 \\
      李浪 & 分组密码 & 10 & 严格 \\
      梁小满 & -- & 1? & -- \\
      林睦纲 & -- & 1? & -- \\
      \bottomrule
    \end{tabular}
    \label{tab:导师介绍1}
  \end{table}
  \footnotesize{\textcolor{blue}{注: 团队情况为--即为新导师。?表示不太确定。}}
\end{frame}

\begin{frame}
  \frametitle{导师介绍 (2/2)}
  \begin{table}[htbp]
    \centering
    \caption{导师介绍(按姓氏首字母排序)}
    \begin{tabular}{cccc}
      \toprule
      导师 & 科研方向 & 团队情况 & 导师性格 \\
      \midrule
      罗泽 & -- & -- & -- \\
      田小梅 & -- & 1 & 和善 \\
      万晓青 & 深度学习+高光谱图像 & 2 & 和蔼 \\
      朱贤友 & 行政岗-托管 & 2 & 和蔼 \\
      赵辉煌 & 深度学习+图像 & 6 & 严谨 \\
      郑光勇 & -- & 1? & -- \\
      郑金华 & 优化算法 & 1 & -- \\
      \bottomrule
    \end{tabular}
    \label{tab:导师介绍2}
  \end{table}
  \footnotesize{\textcolor{blue}{注:郑金华老师,类似联合培养,研二要去湘潭大学。 朱贤友老师,行政岗,托管长沙学院。}}
\end{frame}

\section{小Track}

\begin{frame}
  \frametitle{小Track}
  \begin{block}{复试前联系导师}
    当复试出成绩后,团队人数基本饱和。因此\textcolor{blue}{提前邮件联系}导师,表明自己的诚意和能力,有助于提高进组概率。\textcolor{red}{线下来找导师最优},我能帮忙寻找导师办公室。
  \end{block}

  \begin{alertblock}{提前规划就业方向}
    根据不同方向,调整在论文中时间投入。比如:\textcolor{red}{科研方向}-多些好文章,\textcolor{red}{考公}-少写点文章,多准备公考,etc。因为不同方向并不兼容,\textcolor{red}{不要两头落空}。
  \end{alertblock}
\end{frame}

\begin{frame}
  \frametitle{Q\&A}
  % center show the Thank you page
  \begin{center}
    \Huge\bfseries\textcolor{blue}{谢谢,问答时间!}
  \end{center}
  \footnotesize
\end{frame}

% \begin{frame}
%   \frametitle{参考文献}
%   \bibliographystyle{alpha}
%   \bibliography{../../paper}
% \end{frame}

\end{document}
