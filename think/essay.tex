\documentclass{article}
% Use with [notoc] option to hide table of contents
\usepackage[notoc]{../typesetting/styles/note-zh}
% Default shows table of contents
% \usepackage{../styles/note-zh}
\usepackage{bookmark}

\usepackage{amsthm}
\newtheorem{definition}{Definition}
% set the font family
\setmainfont{Departure Mono}
\setCJKmainfont{Xingkai SC} % Main Chinese font (Songti)
\setCJKsansfont{Heiti TC} % Sans-serif Chinese font (Heiti)
% \setCJKmonofont{Fangsong} % Monospaced Chinese font (Fangsong)

\title{随笔}
\author{isomo}

\begin{document}

\maketitle

\section{前言}

关于写这篇小作文的动机,自己也说不清楚,Maybe是很久没有去梳理一下自己的mind了(这种多字体风格的paper还是第一次排,感觉还可以!!!)。风格属于想到啥就写点啥。不知道是干活干累了,还是对事情失去了新鲜感了,有一种逃避干活的动机涌现,当然不是那种完全讨厌的feeling。这次来一篇去中心化的writing,确保相邻语句去论述不同的事情,突出一个跳跃性,墨守成规的事情干太多了。可能是最近看tech帖子多了,自己也想输出输出,当然我们今天不聊技术,主聊life。

\section{工作OR学习}

当然严格意义上来说,这两者并不突出,但是一段时间内,我们会以其中的一项为主,比如在读研期间,学习时间的占比要比工作多一些。当然如果把学习也归入工作中,那就是另一种说法了,为此我们此处给出一个定义区分一下。


\begin{definition}[工作,学习]
我们从知识或任务的熟悉程度上来判断,如果我们对做一件事是熟悉的,那么我们认为这个工作性质,if这件事是陌生的、不熟悉的,我们认定这是学习。
\end{definition}

\noindent NOTE: 最近学了一下数学和程序编码,感觉数学的定义和定理,证明这些东西,和写代码极其相似。而本身我们很喜欢Coding,So整体写作风格也偏MATH一点点。

此处工作和学习,应该还有一显著的区别,工作是具有明确的交易属性的,是有money可以赚的。学习更像一次高风险的长线交易,时间精力的投入,短时间看不到任何实质性的回报。但是其实此处,我们扩展来讲一下,\textcolor{blue}{具有回报一定是一件好事情?}拥有在某些角度来看,也是对自身行为的一些束缚,甚至可以说是一种负担,这里的判断依据在于不同的角度去评价拥有的事物。如果自己的target就是赚取more and more money,is all right。But 有些时候需要一些情绪价值,如:工作赚钱和自身情绪起到冲突时,choce one?look at me。

其实此处我们发现如果从多个角度去看事物,没有所谓的好坏、对错之分,只有不同目标下的距离之分。当我们的目标是一致的,那么我们判断和行为逻辑就是一致的,具有较好解释性的。但是,但是,人的目标总是多变的,更过一点说,人的目标是善变的,这种变化性像是与生俱来的。

扯的有点远,回到我们工作和学习的问题上来,最近的问题在于自身的目标发生了变化了,应该不是大目标变化了,是对目标的预期距离发生的变化,从对学习的短预期距离拉长了,越来越向短期的交易具有倾向性。体现对学习这件事有点下意识的反感,有点想去打工了,(当然可能是这么个规律,里面的人想出去、外面的人想进来)。

总之最后还是希望自己能balance好工作\&学习。


% 添加参考文献
% \bibliographystyle{plain}
% \bibliography{references}

\end{document}
