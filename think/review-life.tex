\documentclass{article}
% Use with [notoc] option to hide table of contents
\usepackage{../typesetting/styles/note-zh}
% Default shows table of contents
% \usepackage{../styles/note-zh}

% Choose Chinese fonts, if want to change the english font, please use \setmainfont
\setCJKmainfont{Noto Serif CJK SC} % Main Chinese font (Songti)
\setCJKsansfont{Noto Sans CJK SC} % Sans-serif Chinese font (Heiti)
\setCJKmonofont{Noto Sans Mono CJK SC} % Monospaced Chinese font (Fangsong)

\usepackage{bookmark}

\title{Review Life}
\author{isomo}

\begin{document}

\maketitle

\section{引言}

每当自己感觉没啥事情去做的时候,或者说是没有一个明确的目标的时候(迷茫),就想停下来,我们不去考虑未来做什么,而是考虑我们做了什么,回顾一下自己,简单review一下。i.e. 选择文字的表达方式,可能是这个空间的封闭的隐私的,匿名的,怎么说,说什么对身边的人或事情都没有什么影响。So,就有了这一篇文档。

\section{2025 years}

\subsection{05-14}

在过去的两年里,感觉一个很大变化是对于,“好的事情”的认识和理解,或者慢慢有一种对于“好”的另类的追求。
在认知信息圈偏小的过去,只是感觉能去做,就已经很不错了,有了很强的成就感,但当时间线拉长,基础的阶段度过,兴奋的阈值拉高,感觉生活中很多事情都变得\blue{无趣}。

如果某件事情没有比其人做的更好,或者没有使用小的时间去翘动一个大的收益(\blue{大的杠杆}),那么这件事情就变得平常了,进而导致了无趣。自己失去了一种对新事物的好奇心,或者说是对新事物的探索欲望。逐步套入在自己构建的舒适区里面,每天使用自己构建的评价体系去评价自己。做的事情越来越局限,比如我们已经失去了对自己出去玩的动力,感觉出去玩,自己又累又没有实质上的产出,吃力不讨好,感觉不到任何\blue{价值}。

自己变得越来越“\blue{功利主义}”,习惯使用金钱除以时间去规划,这件事情在自己日程的优先级,但是更多时候,自己也不知道有了这钱,一般等价物,也不知道去干嘛,更多时候就是一味的“钱”,去对事情排个序,或者说是对事情的一个评判标准,自己已经失去了\blue{对事情“好”的判断}了。也可以说自己是越来越适应这个世界的某种运行规则了把。

但是几年前的自己,应该还不是这般,会去假想一些美好的品质或者事物,并以此为准则去决定是否去做。我们还是认同,\red{人的一切行为都是他接触的东西的一种map},所以可能是这两年接触的事和人,改变了自己的认知把。\blue{真是越活越成自己以前讨厌的样子了。}

\subsection{05-19}

从过去的快一周时间里面,来看,我们还是没有走出丢失目标的状态中,走出来。一种做什么事情都没有动力的状态,似乎已经成为常态。有人说这是一种gap周,就是我们去休息一段时间,去放空自己。除了做一件事之前,会去考虑一下事情做的意义。但是那有这么多事情,都是有meaning的。可能只是一种自我内耗的做法罢了。为此我们应该对目前的状态进行转化,应该让自己集中起来,把信息流隔断一下。去\blue{主要了解一件事情},平静一下自己的内心。那就去看看数学罗。

% % 添加参考文献
% \bibliographystyle{plain}
% \bibliography{references}

\end{document}
