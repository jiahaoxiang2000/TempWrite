\documentclass{article}
\usepackage{ctex}
\usepackage[a4paper, margin=0.5in]{geometry} % Adjust page margins here
\usepackage{url}
\usepackage{hyperref}
\usepackage{booktabs}

\title{\textit{谈恋爱这件事\footnote{\href{https://github.com/jiahaoxiang2000/TempWrite/blob/master/think/love.tex}{Online Link}}}}

\author{卤蛋2000}
\date{\today}

\begin{document}
\maketitle

\section{背景}
今年是我24岁的一年,回顾这过往的这些年,对于母胎solo的我来说,谈恋爱似乎是一件比较遥远的事情。在对感情的路上,我的起点可能是六年级毕业的那个夏天,那是一个很文静的女孩,至少在我的印象里。可能是时间上有点距离,现在也记不清为什么喜欢她了,但是清楚的记得,躺在一个倾斜的草坪上,望着湛蓝的天空,心里幻想这小学毕业后,在一个中学的日子。那时候的我,对于爱情,只是一个模糊的概念,只是一个美好的幻想,只是一个不知所谓的东西(小屁孩一个,懂啥子嘛)。

不知道是不是自己的幻想起了作用,那个女孩真的和我分到了一个班级,但是我却没有勇气和她说话,只是在心里默默的喜欢着她,加之初中那会接触到了游戏和小说,空余的时间精力都被转移过去了,自卑内向的性格促使了自己没有迈出第一步,使这段故事没有了后续。随后高中学业更加重了点,高中前二年半,都是活在一个人的小世界里,对于爱情的认知全部来自影视作品,对于爱情的向往也只是停留在那些影视的画面里(那时喜欢看古装玄幻剧,男女主那点小故事)。

高中后半年,命运的齿轮开始转动,那天晚上,在学校宿舍,躲在被窝里看了部电影《你的名字》,新海诚的画面、音乐和爱情故事,给我留下了深刻的印象,那时候的我,对于爱情的向往,已经不再是那种幼稚的幻想,而是一种渴望,一种向往,一种憧憬。随后的一次午后,阳光透过楼外的大树与窗户,打到了一位我右后方的女孩的脸上,那一刻开始,我对她有了一种特殊感觉,加之影视作品影响,对爱情的憧憬更加强烈。由于自己的内向胆小的性格,我开始通过QQ给人家发消息,每天都会发一条(这做事打卡的习惯,可能就是那是开始形成的)。每天做梦都会去幻想和她的一些故事,那时候真的很上头,在做了很久的思想斗争后,人家对咱没啥兴趣的情况(那会不会换位思考,脑子里全是自己想法,完全意识不到自己行为对人家的影响),还是给人家写了一个月的信在高考最后的一个月。高考成绩出来后,我去了一个二本的本科,她去复读。大二那年,还参加过百里毅行,走去她的学校见了一面,记得和我说了一句“谢谢给她写的信”,之后就断了联系。

大学期间的前段时间,更多的信息量涌入,同时丢弃了成绩为导向,这个用了十多年的目标,整个人的生活过的很颓废,每天除了吃饱睡足,没有了一点其他要求,过的很佛系。大学的后期,一方面是就业的压力,二是通过网络接触到了计算机这个领域,整个人慢慢有了个目标,“每天写点代码”,之后就是为了这个目标,做了一些事,跨考了研究生。整个大学四年,主动添加联系方式的女孩子不超过五位,更不说接触了(可能是错过了恋爱的黄金期)。

进而转入了最近的这段时光,研一刚入学那会,由于自身为了考研,独处了二年,再次拥抱这个社会,对于爱情的向往更加强烈,但是由于自己的问题,对于爱情的认知,还是停留在那个高中时期。就在这个时刻,遇到了一位很好的女孩,接触多了,又开始了高中时候那一套,先主观带入,她就是我喜欢的那个人,一顿瞎想,一顿幻想,一顿自我感动。看待事情从自己的角度出发更多,对于对方的感受,很多没有意识到(有点下头男那种感觉哈)。在一次线上表白被拒绝后,内心又走向了平静。慢慢开始运动跑步,提升和反思自己。

\section{分析}

从时间上来看,每次喜欢一个人,为了方便分析哈,我们弄一个表~\ref{tab:girl}去做个代称。对于$A$来说,更多的应该是身体上的荷尔蒙开始分泌,慢慢的对异性产生了一种特殊的感觉,对于爱情的认知还停留在一个模糊的概念。对于$B$来说,更多的是对于爱情的模仿,对于爱情的认知已经不再是那种幼稚的幻想,而是具体到一些行动,但是这些行为对于当前情况下是否合适,缺乏必要的判断力。对于$C$来说,是一次对爱情的尝试,但是确实是很生疏,完全没有按正常套路去出牌,随心所欲,显得很是幼稚。但是不可否认的是,每次尝试之后,对自身都是有提升。

\begin{table}[!htb]
    \centering
    \caption{喜欢的女孩代称}
    \label{tab:girl}
    \begin{tabular}{cc}
        \toprule
        代称 & 时期 \\
        \midrule
        $A$    & 小学 \\
        $B$    & 高中 \\
        $C$   & 研一 \\
        \bottomrule
    \end{tabular}
\end{table}

\section{结论}


\end{document}