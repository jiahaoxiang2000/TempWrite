\documentclass{article}
\usepackage{ctex}
\usepackage[a4paper, margin=0.5in]{geometry} % Adjust page margins here
\usepackage{url}
\usepackage{hyperref}
\usepackage{amsthm} % Add the missing package
\usepackage{booktabs}
\usepackage{color}

\newtheorem{theorem}{定理}

\title{\textit{Signification}}

\author{卤蛋2000}
\date{\today}

\begin{document}
\maketitle

\section{引言}

有些时候没去复盘自己的行为,以前复盘自己时,总是会从一件事情的意义出发,如果当前这件事意义不大,那么就会拉低,该事在自己事项里面的优先级,从而对自己的行为做一定的调节。那么在这种运行机制下, 如何去对一件事的意义进行评估呢?这就需要对\textit{signification}进行一定的探讨了。

\textbf{Past.} 如果追溯到蛮久以前,上面的行为调节机制还没形成前,日子过得满随意的,也没什么压力啥子的,脑子里装的只有最近几天的事,对社会的这个模型,构建的很简单。忽然有一天吧,脑子里有了个\textit{idea},自己试着去做了一下,发现做事情吧,需要积累一定的前置储备,从那个时候开始吧,\textit{signification}就等价成了对自己\textit{idea}的积累程度,对\textit{idea}推动越大,就越大,i.e. 当然此时对于\textit{idea}推动多大的评估,多是参考自己或他人经验来进行判断。

\textbf{Issue.} 这种评估方式存在两个问题,1)如果哪天\textit{idea}突然改变了,那么之前的积累就会变得无用,2)如果\textit{idea}的积累是基于自己的经验,那么就会存在\textit{bias},这种\textit{bias}会导致对\textit{signification}的评估不准确。

\section{分析}
就\textit{Issue.1}来说,最近很明显,自己的想法很多,东做一下,西做一下,但是没有一个是持续的,这就导致了自己的\textit{signification}的积累不够,最后很多事情都没有最好,再此过程中也没有感觉到自己做的事情有多有\textit{signification},从而导致对自己行为的正向反馈调节也变少了。而且这种情况下,当某一块任务推不进去的时候,自己就会去做其他的事情,侧面带来一种畏难情绪,下次遇见难点的事,下意识就想溜了溜了。

就\textit{Issue.2}来说,自己的\textit{signification}的评估是基于自己或他人的经验,这就导致了自己的\textit{signification}的评估是不准确的。短时间来看,某些行为的\textit{signification}是很高的,但是时间一长可能是指数增长的,这种经验上的偏差,还是在于经验的样本数量不足以支撑自己的\textit{signification}的评估。如何找到最适合分布的样本,是一个需要解决的问题。

\section{总结}
瞎写了one hour,时间算是打发出去了,就上面的问题还是要提点解决方案,不然感觉啥东西浪费了一样,就\textit{Issue.1},有两个方案,1)目标就不要多,就一个,问题不久解决了,i.e. 自己是个小天才,hhhhh。2)给每个目标分个权重,简单线性拟合一下咯。就\textit{Issue.2},暂时向自己能接触到的,该领域内做的好的学习,多看看他们是怎么评估\textit{signification}的,然后自己再去抄一下作业。今天是7月7号,77好数字,顺口,就这样吧,下次有啥想法,再来补充吧。

\end{document}