\documentclass{article}
\usepackage{ctex}
\usepackage[a4paper, margin=0.5in]{geometry} % Adjust page margins here
\usepackage{url}
\usepackage{hyperref}
\usepackage{amsthm} % Add the missing package
\usepackage{booktabs}
\usepackage{threeparttable}



\newtheorem{theorem}{Theorem}
\newtheorem{assumption}{Assumption}

\title{\textbf{关于谈女朋友这件事}}

\author{卤蛋2000, 爱意随风起}
\date{\today}

\begin{document}
\maketitle

\section{前言}
这篇文章也是随意去写,主要是为了去分析,总结一下,为什么\textbf{小白}(本文男一)在男女关系上,还没开始就失败的这个情况。同时也是扯淡的一篇文章,写的随意,不要当真。为了保护隐私,本文所有的名字都是虚构的。这里我们先对小白的情况做一个简单的介绍,方便读者带入角色,理解其行为。小白属于传统的理工男类型,大学学的土木工程,本科期间,自身游戏打得多,社交圈小,性格内向,缺乏自信,收集以上所有元素,大学单身了四年。毕业后,小白进入了一家建筑设计公司,工作之余,小白开始了自己的恋爱之路。本文用到的名称代号表~\ref{name_replace}如下 (人物排名不分先后,仅用于区分):


\begin{table}[h]
    \caption{重要的代号表}
    \label{name_replace}
    \centering
    \begin{tabular}{cccccc}
        \toprule
        & 小白 & 杰克 & 小红 & 小绿 & 小蓝 \\
        \midrule
        角色 & 男一 & 男二 & 女一 & 女二 & 女三 \\
        \bottomrule
    \end{tabular}
\end{table}


\section{不超过一个月的关系}
\subsection{小红}
遇见小红时,正处于小白工作初期,刚入职的新人小伙,对全新的环境需要一个适应的过程,对外界很多的事情都很有热情,新鲜感还是拉满的,毕竟是一份全新的工作相较与学生时代的小白来说。小红是同期进入公司的新人,机缘之下,被分到了同一个项目组,小红同期已经在几家公司工作过,有着丰富的工作经验,对工作上的事情处理的很好,小白对小红的工作能力很是佩服。开始接触的契机在于项目组分到了一个任务,小红需要小白帮忙,一来二去,二人熟络起来,但此刻二者之间的信息交流仅限于工作。

故事的转折点发生在,小红想要在公司近的地方租房,刚好缺个室友,便邀请小白一起合租,小白考虑了一下,便答应了。此处给足了,二人交流的时间,但是小白的表达能力不强,对小红的好感,言语上没有表达出来。但是行为上,已经开始有了明显的变化,小白开始主动的去帮小红做一些事情,比如说,帮小红买早餐,帮小红洗碗,帮小红拿东西等等。小红对小白的这种行为,也是有所察觉的,好感度也是在增长的。为了更好的,增加二者之间的感情,小白开始组织一些公司的朋友一起玩游戏,希望能在游戏的过程中,拉近二者之间的距离。

此刻的小白,还没意识到,自己犯下了一个mistake,在活动中,引入了一个强力的对手杰克,杰克在外表、谈吐、能力上都比小白要强,在活动的过程中,小红对杰克的好感度也是在增长的。小白在这个时候,还是没有意识到,自己的处境,已经不利,继续的在小红面前表现自己的优点,但是小红对小白的好感度,却是在下降。最终,小红选择了杰克,小白的恋爱之路,又一次的失败了。此后的一段时间内,小白开始修正自己,从外表到一些兴趣爱好的培养,以及提升自我换位思考能力,为下一次的恋爱做准备。

\subsection{小绿}

和小绿的认识,是一件很巧合的事情,有天早上,

\end{document}