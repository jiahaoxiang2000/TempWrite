\documentclass{article}
\usepackage{ctex}
\usepackage[a4paper, margin=0.5in]{geometry} % Adjust page margins here
\usepackage{url}
\usepackage{hyperref}
\usepackage{amsthm} % Add the missing package
\usepackage{booktabs}
\usepackage{threeparttable}



\newtheorem{theorem}{Theorem}
\newtheorem{assumption}{Assumption}

\title{\textbf{关于谈女朋友这件事}}

\author{卤蛋2000, 爱意随风起}
\date{\today}

\begin{document}
\maketitle

\section{前言}
这篇文章也是随意去写,主要是为了去分析,总结一下,为什么\textbf{小白}(本文男一)在男女关系上,还没开始就失败的这个情况。同时也是扯淡的一篇文章,写的随意,不要当真。为了保护隐私,本文所有的名字都是虚构的。这里我们先对小白的情况做一个简单的介绍,方便读者带入角色,理解其行为。小白属于传统的理工男类型,大学学的土木工程,本科期间,自身游戏打得多,社交圈小,性格内向,缺乏自信,收集以上所有元素,大学单身了四年。毕业后,小白进入了一家建筑设计公司,工作之余,小白开始了自己的恋爱之路。本文用到的名称代号表~\ref{name_replace}如下 (人物排名不分先后,仅用于区分):

\begin{table}[h]
    \caption{重要的代号表}
    \label{name_replace}
    \centering
\begin{tabular}{ll}
\toprule
Name & Role \\
\midrule
小白 & 男一 \\
小红 & 女一 \\
小绿 & 女二 \\
小蓝 & 女三 \\
\bottomrule
\end{tabular}

\end{table}

\section{不超过一个月的关系}
\subsection{小红}
遇见小红时,正处于小白工作初期,刚入职的新人小伙,对全新的环境需要一个适应的过程,对外界很多的事情都很有热情,新鲜感还是拉满的,毕竟是一份全新的工作相较与学生时代的小白来说。小红是同期进入公司的新人,机缘之下,被分到了同一个项目组,小红同期已经在几家公司工作过,有着丰富的工作经验,对工作上的事情处理的很好,小白对小红的工作能力很是佩服。开始接触的契机在于项目组分到了一个任务,小红需要小白帮忙,一来二去,二人熟络起来,但此刻二者之间的信息交流仅限于工作。

故事的转折点发生在,



\end{document}