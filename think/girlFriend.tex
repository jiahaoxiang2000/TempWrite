\documentclass{article}
\usepackage{ctex}
\usepackage[a4paper, margin=0.5in]{geometry} % Adjust page margins here
\usepackage{url}
\usepackage{hyperref}
\usepackage{amsthm} % Add the missing package
\usepackage{booktabs}
\usepackage{threeparttable}



\newtheorem{theorem}{Theorem}
\newtheorem{assumption}{Assumption}

\title{\textbf{关于谈女朋友这件事}}

\author{simple, 匿名网友}
\date{\today}

\begin{document}
\maketitle

\section{前言}
这篇文章也是随意去写,主要是为了去分析,总结一下,为什么\textbf{小白}(本文男一)在男女关系上,还没开始就失败的这个情况。同时也是扯淡的一篇文章,写的随意,不要当真。为了保护隐私,本文所有的名字都是虚构的。这里我们先对小白的情况做一个简单的介绍,方便读者带入角色,理解其行为。小白属于传统的理工男类型,大学学的土木工程,本科期间,自身游戏打得多,社交圈小,性格内向,缺乏自信,收集以上所有元素,大学单身了四年。毕业后,小白进入了一家建筑设计公司,工作之余,小白开始了自己的恋爱之路。本文用到的名称代号表~\ref{name_replace}如下 (人物排名不分先后,仅用于区分):


\begin{table}[h]
    \caption{重要的代号表}
    \label{name_replace}
    \centering
    \begin{tabular}{ccccccc}
        \toprule
        & 小白 & 杰克 & 小红 & 小绿 & 小蓝 & 小紫\\
        \midrule
        角色 & 男一 & 男二 & 女一 & 女二 & 女三 & 女四\\
        \bottomrule
    \end{tabular}
\end{table}


\section{不超过一个月的关系}
\subsection{小红}
遇见小红时,正处于小白工作初期,刚入职的新人小伙,对全新的环境需要一个适应的过程,对外界很多的事情都很有热情,新鲜感还是拉满的,毕竟是一份全新的工作相较与学生时代的小白来说。小红是同期进入公司的新人,机缘之下,被分到了同一个项目组,小红同期已经在几家公司工作过,有着丰富的工作经验,对工作上的事情处理的很好,小白对小红的工作能力很是佩服。开始接触的契机在于项目组分到了一个任务,小红需要小白帮忙,一来二去,二人熟络起来,但此刻二者之间的信息交流仅限于工作。

故事的转折点发生在,小红想要在公司近的地方租房,刚好缺个室友,便邀请小白一起合租,小白考虑了一下,便答应了。此处给足了,二人交流的时间,但是小白的表达能力不强,对小红的好感,言语上没有表达出来。但是行为上,已经开始有了明显的变化,小白开始主动的去帮小红做一些事情,比如说,帮小红买早餐,帮小红洗碗,帮小红拿东西等等。小红对小白的这种行为,也是有所察觉的,好感度也是在增长的。为了更好的,增加二者之间的感情,小白开始组织一些公司的朋友一起玩游戏,希望能在游戏的过程中,拉近二者之间的距离。

此刻的小白,还没意识到,自己犯下了一个mistake,在活动中,引入了一个强力的对手杰克,杰克在外表、谈吐、能力上都比小白要强,在活动的过程中,小红对杰克的好感度也是在增长的。小白在这个时候,还是没有意识到,自己的处境,已经不利,继续的在小红面前表现自己的优点,但是小红对小白的好感度,却是在下降。最终,小红选择了杰克,小白的恋爱之路,又一次的失败了。此后的一段时间内,小白开始修正自己,从外表到一些兴趣爱好的培养,以及提升自我换位思考能力,为下一次的恋爱做准备。

\subsection{小绿}

和小绿的认识,是一件很巧合的事情,有天早上,小白在外面散步,碰到一个大学生小绿问路,小白给人家指了路,小绿很感激,便留下了联系方式。小绿对人对事,都有一种说不出的热情,这对于刚经历小挫折的小白来说,是一种很好的补充。小白开始分享自己最近开始学习的一些东西,小绿也是很感兴趣,二人的交流,也是越来越多。小绿对小白的好感度,也是在增长的,小白也是在不断的提升自己。

在接触了一段时间后,转折点出现了,小白公司最近来了一个大项目,小白比较忙,这是刚好小绿邀请小白出去玩,小白也答应了,但是由于工作的原因,小白临时有事,不能去了,小绿对此是失望的,小白也是抱歉。至此,二者之间的关系趋于平静,停止了一段时间的交流。

\subsection{小蓝\&小紫}

小蓝和小紫是公司新入职的两名新人,人美心善,小白对她们的好感度也是很高,但是由于项目组的原因,小白没有太多的时间去和她们交流,只是在公司中,偶尔的交流。为此,小白开始创造一些机会,借助自身入职的一些经验,帮助他们解决一些新入职,可能会遇到的问题,在积累了初步的好感度之后,添加上了联系方式。在随后的几次示好后,未能得到回应,小白也是放弃了。

\section{匿名网友的感想}

随便写写自己的一些感想吧,因为俩件事情产生的对照让我以往的信念产生了动摇,以往我的认知是对妹妹好,给她体贴与关心,在小事上做到无微不至,就可以得到妹妹的心。

小蓝9月8日入职,9月29日左右,我们遇到了她和她的男朋友撑伞在学校闲逛,也就是说短短的20天就在一起了,在时间上击碎了我的认知,在我以往的认知中感情需要很长的时间去培养,去经历,但这个20天的时间是不是有点太短了一点。

小紫,小白坚持示好俩个星期,把心爱的小电驴都低价送给了小紫,而且还请她一起吃饭。最后得到的待遇是朋友圈发照片单独屏蔽了小白。在真心伤击碎了我的认知,扪心而问小白这个人虽然颜值一般,但待人友善真诚,对待妹妹是一片赤忱之心,小心翼翼的处理每一件有关小紫的事情,总想把自己最好的都馈赠给对方。比如每天示好的东西都不让我们去碰,小紫的一切事情仿佛是他的逆鳞一般可以调侃但绝对不能瞎搞。

还有一件小事,小白帮小蓝入职,最后好像请奶茶的事情也不了了之了。我不能理解,我帮同事搬过东西,同事答应请我吃饭虽然最后没有兑现但后面也请我们喝了杯奶茶,也算过了。这俩个对照让我得到的结论就是:妹妹只要对你有感觉,时间,距离,颜值,真心都不是问题,妹妹对你没有感觉,再多的真心对于对方来说这是一种负担。

对于小紫可能小白的方式有问题吧,就是一开始就A上去了显得过于热情导致妹妹退缩了,但这更加说明了妹妹其实对小白没有任何的感觉。因为这俩件事我想到了,狗建南曾经告诉我的,不要着急你,只是缘分还没有到,只是还没有遇到对的那个人。

那么男女之间的交际,到底是什么起决定性因素,对于小白与你我这样的普通人,到底该如何获得妹妹的心?我认为和狗建南说的一样\textbf{顺其自然},让我想起了大学四年暗恋三年而不得的故事,小蓝开学不到一个月,时间不是答案。那么真心呢?小白为他的小红做了很多,亲自做早餐,午餐,一起看电影,送零食,合租室友,坐在相邻的位置,这算是占尽了天时地利人和,最后的结果是选择了别人,放弃了小白。平心而论,小白已经做到了普通直男的极限了,换我绝对没有这个胆量和执行力,这也是我一直敬佩小白的原因之一。

所以最后,找来找去我能找到的答案就是顺其自然,对最简单也最真实的答案,以往我嗤之以鼻,什么狗屁,顺其自然都是些站着说话不腰疼的人说出来的正确的废话。我现在开始考虑其合理性,是否这就是客观现实。妹妹喜不喜欢你,在三天内就确定了,根本不需要太长的时间去培养感情。就是妹妹喜不喜欢你在第一眼或者刚开始的时间就确定了,在之后的时间里你们的交际也就顺着,妹妹不喜欢你这条线一路狂奔。而现实也存在坚持了一年,一个月或者三个月这种,用这么长的时间付出大量的心血和精力追到,也恰恰说明他们在逆势而为。

小白因为在妹妹第一眼看到他时,已经定下了,不可能在一起的发展趋势,之后的印象只会在这个开始的印象上一路发展下去,小红,小蓝,小紫,都是这个原因,因为人家一开始就没看上你,所以之后所做的一切努力,都是在逆转这个势,也就是说妹妹喜不喜欢你,一开始就注定了,剩下的只是在这个基础上做的一些小小努力。再补充一个室友找到女友的例子,也就在微信手机上聊了段时间,也没见说是像小白一样不停地示好的。而且人家还是异地不还是在一起了。举几个顺势而为的例子吧,小白和小绿,人家感冒发烧了一样被小白叫出来一起散步,一起去吃饭。以及我个人的经历,根本没废什么劲,没有正式告白的,或者任何象征性的仪式,你一句我一句话赶话的,就在一起了。

在男女关系上对这个一开始的势起决定性因素的我个人认为是\textbf{外在}。这个外在就是有本身颜值,化妆技术,穿搭衣品综合决定的,还有人品,谈吐,素养,或者某一门特长,这些都是后话了,一开始的外在决定了会不会在一起这个势,而之后作为普通的俩个人,可能会因为双方的人品,素养,或者在某个特殊的时机让这种势产生改变。

但大致来说,就是强势的,有一个特长的,以及外在条件优越的,这三个主要因素影响着我所说的势。基于顺势而为可以推出1)不要在对你没有感觉的人身上投入太多精力,这是事倍功半的做法,2)只能广撒网通过数量来找到真正对你有感觉的人。

还有不知名的谁所说的妹妹喜欢强势的,要从朋友做起,要不排斥你,能约出来一起打游戏,吃饭。我认为这也存在一个顺势而为的问题,我经历过做朋友之后只能是朋友的例子,也就是说那种从朋友发展成男女朋友的,需要一些契机来改变彼此的心态。
  
最后给我的启示就是,\textbf{不可强求},不可以认为自己是个例外接受现实,找女友只能顺势而为,或者叫顺其自然。除此之外的很多方法都是需要投入大量的时间精力与心思的,这就又牵扯到另一个问题,这个妹妹值不值得你投入那么多时间精力去追求,你追求得来,妹妹是真的喜欢还是因为你的行为暂时而被感动。

这种逆势而为的行为一开始,就会把自己放在很被动很卑微的位置上,这是极其不健康的相处模式。说这么多并不是想怪谁,因为你的喜欢与他人无关,他人不需要为你的真心背负什么枷锁。不要有任何的幻想,只要妹妹对你不主动那就是对你没感觉,而何为不主动,你不找她,她绝对不会主动找你。而追是需要付出很多精力,让她对你有感觉,这是一件很难办到的事情,是一件逆天道而行的事情。

最后的最后,我再讨论下颜值的问题,可悲的是我们都是凡人,没有办法绕过颜值直接对一个人品很好地人产生感情。就像你告诉我,你给我介绍一个美女给我认识我会很期待,但你告诉我你要给介绍一个好人,我就没有太高的兴致。就是你的眼光太高,总喜欢一些好看的,这不是你的实力所能够到的,不要把颜值作为第一标准。一切顺其自然吧,找个互相看着顺眼的,在一起这样的事情就是很简单的事情。

\end{document}