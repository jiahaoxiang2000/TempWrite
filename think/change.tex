\documentclass{article}
\usepackage{ctex}
\usepackage[a4paper, margin=0.5in]{geometry} % Adjust page margins here
\usepackage{url}
\usepackage{hyperref}
\usepackage{amsthm} % Add the missing package
\usepackage{booktabs}

\newtheorem{theorem}{Theorem}
\newtheorem{assumption}{Assumption}

\title{\textit{Change: 2021To2024}}

\author{ludan2000}
\date{\today}

\begin{document}
\maketitle
\section{Motivation}
其实很早之前就,一直想去写一下最近自己的变化,回看、总结过去,为了更好的规划一下未来吧。但是一直没有动力,也没有时间,也没有想法。但是最近,我发现自己的状态有点不对劲,所以就想着,还是写一下吧。看能不能把自己的状态调整一下。看能不能,把一些好的东西,留下来,把一些不好的东西,去掉。所以,就有了这篇文章。

\section{Change}
其实自己变化的还有点多,一时间也不知道怎么组织呢,所以就想着,还是按照时间顺序来写吧。这样,也能够更好的回忆起来,也能够更好的总结一下。
\subsection{2021}
时间这东西还是真的快,2021那年正好是本科毕业那年,大家为了各自的目标而努力着。不知道是自己笨还是懒,可能是综合的,想找个“轻松”一点活,自己选择了转行,从事计算机相关的工作。当时想着,只要是编程相关的工作就行,不管是前端还是后端,不管是Java还是Python,只要是编程就行。所以,就开始了自己的转行之路。那时候看什么都是好奇的,什么技术都想去试一下,看一下他的内部运行逻辑,i.e., 像小孩子刚接触新玩具一样。所以,那时候,自己的状态还是挺好的,每天都有很多的动力,很多的激情。但是,也正是因为这种状态,导致自己的学习方向不够明确,学习的内容不够系统,所以,也就没有什么深度。去找的\textbf{job}也是一些小公司,一些小项目,所以,也就没有什么发展。为了解决这个深度问题,自己就开始了自己的考研之旅吧。
\subsubsection*{conclusion}
主要的改变是,对未来的日子有了个\textbf{明确的大目标},也就是转行去计算机。其次开始把\textbf{长期}一点事情放入自己的考量中,比如为了加点知识深度去考研。

\subsection{2022}
这年是很快的一年,上次花了6个月准备了第一次考研,结果是符合预期的,没有考上。因为在备考期间自己的一些习惯,严重影响了自己的学习效率。记得那时候考前晚上,搁哪总结2021考研的问题呢,考前就知道这次肯定上不了,一轮复习都没搞完,拖延症拉满了是。但是好的是,自己意识到了这些习惯会导致自己预期与实际脱节嘛,so,自己一顿总结输出,哪些点需要改,怎么改。为了改行这个大目标,搞了一个考研的小目标,出了成绩之后,i.e.,考了100多分,很惨,但符合预期哈,开始那2个月吧,框框一顿学呀。然后老毛病又来了,心思散了,把一些时间分到计算机领域的开荒里面去了,整了好几个web的个人博客,租了几个便宜服务器,开始去写点代码,那一年写了302次commit,尝试蛮多新玩意。那么代价就来了,考前6个月,就感觉自己如果考数一,408这套复习是搞不定,那就将目标,稳一点,找个上线就要的学校,这不压力就小了。最后考试的时候,英语考完就感觉到了,这次有个地方收我了,i.e., 哈哈哈哈,阅读理解我全看明白了,好像,那些英文的技术文档没白看,结果20个只错了3个。

\subsubsection*{conclusion}
主要的改变:明白了对于自己这种非天赋型选手,\textbf{坚持}可能是区分自己和其他人价值的主要手段,不多干活,搞不过其他对手丫。这一年我应该是一个人在小房间里自习了一年,学会一点\textbf{和自己相处}吧。记得那个时候,每天早晚自己都会去散个步,安排下自己明天的事情。

\subsection{2023}
在忙乎了快一年半之后,终于找到了个收留我的地方,找了个能读书学习的地方。进了实验室,遇见了现在的导师,带我进入了一个全新的计算机小领域,接触一套全新的做事方式吧,高效率的做事风格,i.e., 这个感觉是做学术种普遍有的,怎么把自己的东西,简洁明了的表达出来,同时保证每句话的准确性。这对我之后的行为模式产生了很大的影响。当然又重新步入了校园,朋友是少不了了,又又交了很多新的朋友,各种搭子搞起来了,i.e,主要上一年,给自己关了一年,终于放出来了。学校毕业需要一篇小论文,所以主要目标成了搞论文,为此学了蛮多新的东东,最后赶在年底把论文写完了,i.e., 运气好次年中了。同年为了提升自己的形象,减了一年重,从189到了140。
\subsubsection*{conclusion}
主要改变:将自己的目标减少到了三个以内,去\textbf{提升自身的效率}, 收获到蛮多以前没接触的东东,减肥40斤丫,新的搭子朋友,突然有点\textbf{成就感},同时在b站开始输出自己的一些见解丫。


\end{document}