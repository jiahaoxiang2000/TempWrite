\documentclass{article}

\usepackage[a4paper]{geometry} % Adjust page margins here
\usepackage{url}
\usepackage{hyperref}
\usepackage{amsthm} % Add the missing package
\usepackage{amsmath}
\usepackage{amssymb}
\usepackage{booktabs}

\newtheorem{theorem}{Theorem}
\newtheorem{assumption}{Assumption}
\newtheorem{definition}{Definition}
\newtheorem{proposition}{Proposition}

\title{\textit{Love}}

\author{Jiahao Xiang}
\date{\today}

\begin{document}
\maketitle

\section{Introduction}
Love is a complex emotion, often characterized by behaviors that can be difficult to understand. We want to more deeply understand love by modeling on the mathematical.

\section{Modeling}
\subsection{Basic Definitions}

\begin{definition}
    Love is a relationship between two people, denoted as $a$ and $b$, where $a, b \in P$. We define the love between $a$ and $b$ as:
    \[
        L(a, b) = \{(a, b) \mid a, b \in P \text{ and } L(a, b) \text{ holds}\}
    \]
    Here, $P$ represents the set of all people, and $L(a, b)$ is a predicate that holds true if and only if there is love between $a$ and $b$.
\end{definition}


\begin{proposition}
    Love is not a symmetric relation, i.e., $L(a,b) \neq L(b,a)$.
\end{proposition}

\begin{proof}
    To prove that love is not a symmetric relation, we need to show that there exists at least one pair $(a, b)$ such that $L(a, b)$ holds but $L(b, a)$ does not hold.

    Consider two individuals $a$ and $b$. Suppose $a$ loves $b$, which we denote as $L(a, b)$. However, this does not necessarily imply that $b$ loves $a$. In other words, $L(b, a)$ may not hold.

    For example, let $a$ be an individual who has expressed love towards $b$, but $b$ does not reciprocate this feeling. In this case, $L(a, b)$ is true, but $L(b, a)$ is false. Therefore, $L(a, b) \neq L(b, a)$.

    This example demonstrates that love is not a symmetric relation.
\end{proof}

\subsection{Evaluation Love Intensity}

\begin{definition}
    The intensity of love between two individuals $a$ and $b$ is denoted as $I(a, b)$, where $I(a, b) \in \mathbb{R}$. The intensity of love can be evaluated based on the following factors:
 
\end{definition}


\end{document}